%%%%%%%%%%%%%%%%%%%%% Librería para el cv %%%%%%%%%%%%%%%%%%%%%
\documentclass[letterpaper]{cvclass}
%%%%%%%%%%%%%%%%%%%%%%%%%%%%%%%%%%%%%%%%%%%%%%%%%%%%%%%%%%%%%%%

%%%%%%%%%%%%%%%%%%%% Información personal %%%%%%%%%%%%%%%%%%%%%
\cvprofilepic{src/images/foto.jpg} % Foto de perfil

\cvname{Gabriel Andrés Aguirre} % Tu nombre
\cvjobtitle{Estudiante avanzado de Ingeniería en Mecatrónica} % Titulo o trabajo

\cvdate{1 de Marzo de 1999} % Fecha de nacimiento
\cvaddress{Concordia, Entre Ríos} % Residencia
\cvnumberphone{+5493454014944} % Número de teléfono
\cvsite{https://github.com/GabiAndi} % Sitio web personal
\cvmail{gabiandiagui@gmail.com} % Correo electrónico
%%%%%%%%%%%%%%%%%%%%%%%%%%%%%%%%%%%%%%%%%%%%%%%%%%%%%%%%%%%%%%%

%%%%%%%%%%%%%%%%%%%% Inicio del documento %%%%%%%%%%%%%%%%%%%%%
\begin{document}

% Sobre mi
\aboutme{Apasionado por la electrónica y programación de bajo nivel. Me gusta investigar y aprender sobre nuevas tecnologías, ademas de ampliar mis conocimentos en sistemas de control y automatización.} 

% Sección de habilidades, con escala del 0 al 6 y decimales
% \skills{{pursuer of rabbits/5.8},{good manners/4},{outgoing/4.3},{polite/4},{Java/0.01}}
\skills{{Programación de sistemas/4.4},{Sistemas operativos/3.6},{Software de ofimática/2.6},{CAD y CAM/1.5},{Electrónica/3.2},{Mecánica/2},{Controles industriales/4.0}}

% Texto añadido sobre las habilidades personales
% \skillstext{{lovely/4},{narcissistic/3}}
\skillstext{}

%%%%%%%%%%%%%%%%%%%%%%% Primera página %%%%%%%%%%%%%%%%%%%%%%%%
\makeprofile 

\section{Intereses}

Siempre aprender sobre tecnologías emergentes y especializarme el la creación de dispositivos nuevos, o soluciones innovadoras.

\section{Educación}

% Lista con descripción
%\twentyitem{<Fecha>}{<Titulo>}{<Estado>}{<Descripcion>}
\begin{twenty}
	\twentyitem{2016}{Bachiller con orientación en Ciencias Naturales}{Completo}{Colegio Nuestra señora de los Angeles}
	\twentyitem{2017-2022}{Estudiante de Ingeniería en Mecatrónica}{En proceso}{Facultad de Ciencias de la Alimentación de la UNER}
\end{twenty}

\section{Experiencia}

% Lista corta sin descripción
%\twentyitemshort{<Fecha>}{<Titulo/Descripcion>}
\begin{twentyshort}
	\twentyitemshort{2016}{Pasante en \href{http://www.dilfer.com.ar/}{Dilfer S.A.}, Sector de infraestructura.}
	\twentyitemshort{2018-2020}{Tutor par en la \href{https://www.fcal.uner.edu.ar/}{Facultad de Ciencias de la Alimentación de la UNER}.}
	\twentyitemshort{2021}{Pasante en \href{https://defymotion.com.ar/}{DEFYMOTION S.S.}, Sector de electrónica y programación.}
\end{twentyshort}

\section{Proyectos}

% Lista corta sin descripción
%\twentyitemshort{<Fecha>}{<Titulo/Descripcion>}
\begin{twentyshort}
	\twentyitemshort{2021}{\href{https://www.fcal.uner.edu.ar/prototipo-listo-microbiologia-e-innovacion-en-la-docencia-en-camino/}{UVC, dispositivo para la eliminación de microorganismos por rayos UV tipo C para el protocolo de vuelta a clases en la universidad.}}
	\twentyitemshort{2021}{Control del sistema de aireación en planta de tratamiento de aguas residuales.}
\end{twentyshort}

\section{Cursos y formación}

% Lista con descripción
%\twentyitem{<Fecha>}{<Titulo>}{<Estado>}{<Descripcion>}
\begin{twenty}
	\twentyitem{2018}{Curso sobre el uso del motor gráfico Unreal Engine.}{Completo}{Orientado a proyectos de Ingeniería.}
	\twentyitem{2019}{Curso sobre el uso del motor gráfico Unity.}{Completo}{Orientado a la creación de contenido interactivo.}
	\twentyitem{2020}{Curso para el desarrollo Bare Metal.}{Completo}{Programación de microprocesadores Cortex-M y Cortex-A.}
	\twentyitem{2021}{Crédito de Diseño y confeción de planos eléctricos.}{Completo}{Planos eléctricos para instalaciones industriales.}
	\twentyitem{2021}{Crédito de diseño y manufactura de PCBs.}{Completo}{Buenas prácticas a la hora del diseño y la manufactura de PCBs.}
	\twentyitem{2021}{Crédito sobre gestión del mantenimiento.}{Completo}{Orientado al mantenimiento de equipos industriales.}
	\twentyitem{2021}{Crédito de industria 4.0.}{Completo}{Implementación de tecnologías 4.0 para industria.}
	\twentyitem{2021}{Crédito de programación en VHDL.}{Completo}{Orientado a la programación de FPGA.}
	\twentyitem{2022}{Curso para el desarrollo de aplicaciones en Qt.}{Completo}{Creación de APPs multiplataforma con Qt Quick.}
\end{twenty}

%\section{Referencias}

%Ing. Daniel Gil, Grupo Dilfer (Concordia). Teléfono: +5493454108910. E-mail: \href{mailto: danielgil@dilfer.com.ar}{danielgil@dilfer.com.ar}.

%%%%%%%%%%%%%%%%%%%%%%% Segunda página %%%%%%%%%%%%%%%%%%%%%%%%
%\newpage

%\newgeometry{left=2.5cm,top=1cm,right=1cm,bottom=1cm,nohead,nofoot}

%\section{Mas info}

%Mas información

\end{document}
%%%%%%%%%%%%%%%%%%%%%%%%%%%%%%%%%%%%%%%%%%%%%%%%%%%%%%%%%%%%%%%